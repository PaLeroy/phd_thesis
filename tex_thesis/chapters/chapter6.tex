\chapter{Competition}\label{ch:competition}
\begin{chapter_outline}

This chapter introduces basic concepts and literature stories of competitive and general-sum settings of MARL.
This is the third background chapter of this manuscript.
Section \ref{sec:ch6_intro} introduces this third part of the manuscript, wrapping up the two MARL settings introduced earlier.
The solution of each agent maximising its return can suffer from different problems, and we present other types of solutions in Section \ref{sec:ch6_solutions}.
Section \ref{sec:ch6_history} presents historical methods to learn in non-cooperative environments, from game theory to adversarial search and MARL.
We conclude this chapter in Section \ref{sec:ch6_algo} by defining methods to solve competition and general-sum settings.

\end{chapter_outline}

\section{Introduction}\label{sec:ch6_intro}
In Part \ref{part:background} of this manuscript, we present the general framework for MARL and its different settings.
In Part \ref{part:coop}, we focus on the cooperative setting, where all agents receive the same reward.
In this third part of the manuscript, we focus on a different setting where two teams of agents compete against each other.
This chapter provides an overview of MARL in the competition and general-sum setting.
The next one then presents the two-team competition setting and conducted experiments to highlight how to train teams by combining the methods defined in Part \ref{part:coop} and the methods to train agents to compete presented in this chapter.

One problem of MARL is that agents may never stop adapting their policies to the new policies of other agents who are also adapting to changes since they are also learning.
This never-ending adaptation cycle led many researchers to define equilibria, achieved when agents do not want to change their policy.
Moreover, when agents receive a common reward, like in Dec-POMDP, deciding which of two joint policies is the best is straightforward.
However, this becomes less obvious when agents receive different rewards, like in competitive and general-sum settings, where the change in the return of agents may not be fair between two joint policies, equilibria or not.
As introduced in previous chapters, finding and selecting an equilibrium represents one of the main challenges of MARL.
Section \ref{sec:ch6_solutions} discusses different solutions in MARL to achieve an equilibrium and rank policies.

Once these solutions are defined, Section \ref{sec:ch6_history} presents an overview of the history of acting in multi-agent systems.
As will be discussed, game theory defines solution concepts, and we cover some of its foundations, including historical RL and planning methods.
Finally, Section \ref{sec:ch6_algo} discusses self-play and population-based training, two learning scenarios allowing agents to learn to act in complex games.

\section{Solutions}\label{sec:ch6_solutions}
The solution to an MDP is the policy that maximises the agent's expected return.
While finding the optimal policy may be challenging, deciding which of two policies is better, given the expected return, is simple.
In a POSG, the return of each agent is $G_0^{a_i} = \mathbb{E}_{\mathbf{\pi}}\left[ \sum_{t=0}^{T-1} \gamma^t r^{a_i}_t \right]$, also shortened $G^i(\mathbf{\pi})$ in the following.
Choosing between two joint policies is as straightforward in a Dec-POMDP as in SARL because agents all receive the same reward, and their return changes identically.
However, in other settings, when agents receive different rewards, it can become challenging to determine which joint policy is a better solution to the POSG.
This is because the change in the return of each agent will likely be different.
Moreover, maximising each agent's return may lead to never-ending learning, where each agent adapts to the others.
Therefore, alternative solutions to games have been defined, and this section provides some.
\cite{marl-book} defines MARL as a pair composed of an environment and a solution concept.
While this section defines some, a thorough list can be found in \citep{marl-book}.

One of the solutions is the best response.
It evaluates the policy of an agent $a_i$ by considering that the policies of all other agents $\mathbf{\pi^{-i}}$ are fixed.
The best response policy is in the set of $\argmax_{\pi^{i}} G^i (\pi^{i}, \mathbf{\pi^{-i}})$ and may not be unique.
It is indeed the solution that provides the optimal return when other agents' policies are fixed.
It can be used to iteratively compute a solution for all agents by computing the best response policy of one agent after the others, as will be shown later.

Another well-known solution is the minimax solution, which we define in a two-player zero-sum game.
In such a game, the agent's reward can be modelled as the negative of the other's, so $r_t^{a_i} = -r_t^{a_{-i}}$ and $G^i(\mathbf{\pi}) = - G^{-i}(\mathbf{\pi})$.
A joint policy $\mathbf{\pi} = (\pi^i, \pi^{-i})$ is a minimax solution if $G^i(\mathbf{\pi})=\max_{\pi^i} \min_{\pi^j} G^i(\pi^i, \pi^j)$.
Again, several minimax solutions can exist, but their returns are equal, called the game's minimax value.
In other words, the minimax value can be achieved from a given state, typically the initial one, given that both players play optimally \citep{russel2010}.
Regarding best responses defined previously, one can interpret the minimax joint policy as the best response of one agent to the best response policy of the other \citep{marl-book}.

Extending this concept of mutual best responses to general-sum games with two or more agents can be the definition of the Nash equilibrium \citep{marl-book}.
This equilibrium is achieved when no agent benefits from changing its current policy.
A joint policy $\mathbf{\pi}$ is a Nash equilibrium if $\forall a_i$, any $\pi'_i$ is such that $G^i(\pi'_i , \mathbf{\pi^{-i}}) \le G^i(\mathbf{\pi})$ \citep{nash1950equilibrium}.
For some environments, only stochastic policies can achieve a Nash equilibrium.
For the rock-paper-scissor game, one example of stochastic policies achieving a Nash equilibrium is to take each action with a uniform probability.
Moreover, a Nash equilibrium is not necessarily unique and has different expected return values for each agent.
This highlights the challenge of selecting the optimal equilibrium because the difference in returns between equilibria is not always the same between agents.

These three solutions have been demonstrated in an SG, and we refer the reader to the book of \cite{marl-book}.
They also discuss the complexity of computing an equilibrium and explain that Nash equilibrium cannot be computed in polynomial time in the general case.
Variants and extensions of the Nash equilibrium also exist.
The $\epsilon$-Nash equilibrium allows a surrounding region to improve flexibility and reduce the strict Nash equilibrium.
Until now, we have silently considered that agents take actions independently.
However, this is not always the case.
Indeed, agents may have correlated policies leading to the correlated equilibrium, also called coarse equilibrium.
This equilibrium generalises the Nash equilibrium to correlated policies.
Both are fully covered by \cite{marl-book}, who also conclude the definitions of equilibrium by highlighting their three main limitations: the possible sub-optimality, the non-uniqueness, and the incompleteness of these equilibrium solutions.
The latter means that the equilibrium solution does not provide specific actions to guide players back to the equilibrium path once they deviate from it.

A non-equilibrium solution is a Pareto optimal joint policy.
It comes from the multi-objective optimisation literature \citep{ehrgott2012vilfredo}.
A joint-policy $\mathbf{\pi}$ Pareto dominates an other one $\mathbf{\pi'}$ if $G^i(\mathbf{\pi}) \ge G^i(\mathbf{\pi'})\forall i$ but for one agent $a_i$, $G^i(\mathbf{\pi}) > G^i(\mathbf{\pi'})$.
In other words, a joint policy is Pareto-dominated if one agent can achieve better returns without reducing the returns of others.
A Pareto optimal joint policy is easily defined as not Pareto-dominated by any other.
The difference between Nash equilibrium is that here, no agent can improve its return without worsening one of the others.
Meanwhile, the Nash equilibrium considers a single agent to change without becoming worse.
This Pareto optimality combined with equilibrium allows the solution to be improved.
Specifically, it reduces the number of equilibriums considered because it seems evident that a Pareto optimal equilibrium is better than a non-Pareto one.
However, there is often no unique Pareto-optimal joint policy.
More importantly, a joint policy can be Pareto-optimal without being an equilibrium one.
Finally, Pareto optimality does not consider how the return is distributed amongst agents, leading to the notion of social welfare and fairness detailed in \citep{marl-book}.

\section{History}\label{sec:ch6_history}
Game theory (GT) \citep{von1947theory} is as much the foundation of MARL as is RL \citep{Nowe2012GTMARL, marl-book}.
The solution concepts presented in Section \ref{sec:ch6_solutions} and the classification in three types of multi-agent settings are results from game theory.
GT with RL has a long history, ranging from normal-form games to repeated normal-form games, and POSG is the complex extension of these settings.
In the following, we present the premise of binding GT and RL and cover methods that have led to what exists now to try to solve POSG.
We finish with planning methods dedicated to the extensive-form games that contributed to POSG's progress.
This section presents some methods at the foundation of MARL to highlight foundations from dynamic programming.
These methods are described in \citep{marl-book} and \citep{russel2010}, already exploited in this manuscript, in addition to the chapter ``Game theory and Multi-agent RL'' \citep{Nowe2012GTMARL} presenting others.

In non-repeated normal-form games, finding the minimax solution with linear programming \citep{marl-book} is possible.
Game theory with RL also started in normal-form games.
For another example, we can cite the study on independent Q-Learning in cooperative normal-form games by \cite{claus1998dynamics}.
Many methods have also been proposed for SG.
Learning the optimal value function for all states in an SG is possible using dynamic programming.
Value iteration \citep{stochasticGames} is an example of such a method, with the same name as its homologue in MDP \citep{sutton2018reinforcement}, despite being slightly different in concept.
As in SARL, this method requires knowledge of the environment model and has the same limitations.

A class of methods involves modelling other agents to predict their policy.
This is called agent modelling, which means that their policy is approximated from the observed past actions.
Once other agents' policies are modelled, the best response against their approximated policy can be computed.
This is done in fictitious play \citep{brown1951iterative} for the normal-form game or more recently in Poker \citep{pmlr-v37-heinrich15} 
Extensions consider additionally learning how the opponent will update its strategy \citep{he2016opponent, foerster2017lola}.

There is also a distinct class of method called joint action learning.
These methods directly learn $n$ state-joint-action value functions $Q_i(s, \mathbf{u})$ to learn equilibrium Q value iteratively with a temporal difference.
However, these methods require assumptions on the type of equilibrium agent aim.
This leads to different algorithms, such as Minimax Q-learning \citep{MarkovGames}, Nash Q-learning \citep{hu2003nash} and Correlated Q-learning \citep{greenwald2003correlated}.
These algorithms have some limitations in the type of equilibrium they can achieve \citep{marl-book}.
The joint action learning class of the method forces all agents to follow the same equilibrium desire, or at least learn, considering that they follow this same desire of equilibrium.

Many more details and methods can be found in \citep{marl-book}, \citep{Nowe2012GTMARL} and \citep{russel2010}.
We presented an overview of existing methods to provide a bit of history.
In the next section, we focus on self-play and population-based methods that motivated the work presented in Chapter \ref{ch:2teams}.
However, we conclude this section with an overview of planning methods in multi-agent systems.

Section \ref{sec:ch2_model_based_vs_model_free} defines planning as computing a local solution using the environment model.
Planning has a long history in multi-agent systems, especially in fully observable two-player zero-sum games with turn-taking actions.
These games are called extensive-form games, which differ from repeated normal-form games because agents do not act simultaneously \citep{Nowe2012GTMARL}.
Famous examples include Chess and Go.
In such board games, agents take action, modifying the state of the board that is observable by both.
Both obtain zero rewards except at the end of the game.
If one wins, it obtains a reward equal to 1; if it is a draw, both receive a reward equal to 1/2.
The sum of all rewards is thus always 1.
One family of planning methods to play this game is adversarial search \citep{russel2010}.

Adversarial search is an extension of classical tree search, where a tree representing all possible games is built by considering all possible successive actions of the agent.
In such a tree, each node represents a state, and each edge represents an action. 
A path from the root to a leaf thus represents one possible game.
The best solution is, therefore, the path toward the leaf providing the best outcomes.
Growing such a tree can become intractable when the number of possible games increases.
This leads to methods that build only a subpart rather than the entire tree using information during the search.
This information is typically given by an evaluation function approximating the best achievable outcomes from a given node.
A well-known informed tree search method is A*.

Back to adversarial search, growing the tree of all possibilities is possible.
However, one needs to consider the presence of other agents, especially the uncertainty regarding their future actions.
This motivates the need for a search algorithm to be executed for every action.
In Section \ref{sec:ch6_solutions}, we presented the minimax solution providing the minimax value if both players play optimally.
The minimax algorithm is an adversarial search algorithm that computes this minimax value from each tree node.
Therefore, the optimal action is to select the action with the maximum minimax value or the minimum for the other player.
However, despite being able to prune such a tree and not compute all values with the alpha-beta minimax algorithm, problems still arise as the tree's size increases.
A solution is thus to define a criterion to stop expanding the tree at a given node and approximate the minimax value at this node.
However, this approximation cannot always be perfect and could lead to suboptimal choices in complex games.
The reader should now understand why a game's complexity is linked to the number of possible games.

This leads to Monte Carlo tree search (MCTS), in which states are evaluated by playing games with random actions instead of an evaluation function.
This allows the agent to estimate the win rate achievable from a state without knowledge.
In MCTS, the tree is expanded iteratively by selecting nodes to evaluate by random games.
The underlying idea is not to explore the tree arbitrarily but to focus on nodes that frequently provide good results.
Simple rules define the selection and expansion process of nodes.
Random games are played from intermediate nodes, not always the root, whose direct parent have already been used to start a random game.
For each expanded node, the win rate associated with the action taken to reach it and the number of times this node has been selected are stored.
These two quantities drive the selection process, highlighting the exploration/exploitation dilemma to play games from unexplored nodes while maintaining certainty of good ones.

AlphaGo \citep{silver2016mastering} allowed an agent to beat the best humans by modifying the traditional MCTS to benefit from the learning capabilities of neural networks.
Instead of playing random games to approximate win rates and probabilities of taking actions, two neural networks approximate them.
The MCTS searches are guided by neural networks, which play games that are then used to train the neural network.
These networks are initially trained with a dataset of human games before using games generated by letting the agent play against itself.
AlphaGo Zero \citep{silver2017mastering} is the extension without pre-training the networks with human data and other minor changes.
They have been generalised to AlphaZero \citep{silver2018general}, which also plays Go and Shogi.
Later, such tree search method has been extended to MuZero, which learns the model of any game to play it  \citep{schrittwieser2020mastering}.
MCTS and these two extensions are part of the self-play class of algorithms, training the same strategy for all agents in the environment.
The following section presents works that combine model-free RL and self-play.
The extension considers population-based training, where multiple agents train by playing against each other.

\section{Self-play and population-based training}\label{sec:ch6_algo}
Its name implicitly defines self-play as methods that train an agent by making it play against itself.
In the previous section, we mentioned the model-based methods that have been generalised to AlphaZero \citep{silver2018general}.
However, self-play can also be exploited with model-free methods.
One of the first methods combining model-free RL with neural network and self-play is TD-Gammon, which plays backgammon, starting from random network parameters to achieve a master level \citep{tesauro1994td}.
Later, the success of modern RL methods allowed the achievement of human-level performance in other games, such as in Stratego \citep{stratego}.
Self-play has also been used with more than two competing agents in mixed cooperative-competitive environments, like the framework introduced in Chapter \ref{ch:2teams}, such as Hide and Seek \citep{baker2019emergent} or open AI five \citep{openai2019dota}.
In Hide and Seek, they trained agents in self-play with PPO and a centralised critic with access to the state, a CTDE method.
They demonstrated that the auto-curriculum provided by self-play, in addition to learning successive strategies, allows agents to understand the environment better and perform better on new tasks than agents trained with the classical unsupervised RL method, where agents learn to explore the environment.
Self-play also shown benefits for autonomous driving \citep{cornelisse2024humancompatible}.

In some, the self-play agent sometimes plays against an old version of itself.
This allows it to be resilient against diverse strategies, not generalise, and not forget how to play them.
An essential requirement of self-play is the policy that the agent learns should be able to act in place of other agents.
When agents are heterogenous, typically in their uncertainty about the state or in their action space, self-play may not be possible.
A solution is to train a population of policies for each agent \citep{jaderberg2017population}.
Moreover, in self-play, when agents play against a past version of themselves, it is considered a particular case of population-based training because several policies are kept to play against.

There are several examples of games where human-level performance has been achieved with population-based training.
We can think of Quake III Arena in Capture the Flag mode \citep{jaderberg2019human}, StarCraft \citep{vinyals2019grandmaster}.
In these two, self-play is still possible because agents are the same.
Still, population-based training allows for solving complex games because, on the one hand, it updates all policies from the population instead of just one.
On the other hand, there are many different strategies in such games.
Indeed, the classical process of population-based training starts with several policies, which are evaluated and updated as they play games against each other.
The population may be modified by increasing its size, duplicating some of its individuals, or freezing others.
Maybe the most successful method when training a population is to use policy space response oracles (PSRO) \citep{NIPS2017_3323fe11, Muller2020A}.
In short, PSRO reframes population-based training as a normal-form metagame in which actions are policies.
It computes the probabilities of selecting each policy in each population of agents to achieve an equilibrium.
From these computed distributions, it is possible to create new best-response policies to be added to the population.
The complexity of such a method is immense because, on the one hand, the distributions of actual strategies to achieve equilibrium should be established.
On the other hand, the best responses to such distribution should be computed by training new agents.