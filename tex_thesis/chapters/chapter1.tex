\chapter{Introduction}\label{ch:introduction}

We make decisions every day.
Let us build some intuition about this decision-making process through the following example.
Consider yourself driving a car and arriving at a crossroads.
In front of you, there is a traffic sign.
You predict it is a stop sign based on your perception and knowledge.
This prediction involves reasoning and has no impact on your environment.
As suggested by the traffic sign, you then take the action to stop the car.
This also involves reasoning but impacts your environment.
Indeed, you push on the brakes, and the car decelerates until it stops.

These decisions are based on what you learned, even if some can be made without knowledge.
Indeed, you probably studied the traffic signs to know it was a stop one.
You practised your driving skills to obtain your driver's license by trial and error, but hopefully also with the help of a supervisor.
You did acquire these skills by learning.
You learn by supervision, having examples of decisions and feedback to highlight whether some were right or wrong.
However, you also learn by interacting with and testing our environment.
Many may recall a child tossing objects from a table, discovering gravity in action, possibly motivating the quote from \cite{sutton2018reinforcement} ``The idea that we learn by interacting with our environment is probably the first to occur to us when we think about nature of learning''.
In summary, you learn by yourself and with the help of others.
Nevertheless, this manuscript is not about how humans learn to make decisions.
It is about how computers can learn to make decisions.

When a computer makes decisions, it is called artificial intelligence (AI).
However, AI is not only about computers making decisions.
Nowadays, AI has many definitions, and we will provide some before focusing on the one of interest.
In their book, \cite{russel2010} rely on several definitions to propose a classification of AI into four categories.
They classify the definitions of AI as making computers either think like humans or think rationally or act like humans or act rationally.
Computers thinking like humans is the field of cognitive sciences where theories of the human mind are built, and computers thinking rationally is the field of logic, based on knowledge representation and inspired by the syllogisms invented by Aristotle.
This manuscript does not address these two types of AI.
It also does not address acting humanly, which we justify hereafter, but focuses on AI acting rationally.
Rationality means that it maximises the measure of the performance of the decision.
Therefore, these types of AI are built to make decisions to achieve a predefined goal.
We commonly define anything making decisions as an agent.

When considering AI acting like humans, a well-known idea coming up is the Turing test.
An agent succeeds in the Turing test if, during a written conversation with it, the human cannot tell whether its interlocutor is an AI.
This test has been extended to include video signals or physical interaction to test more complex problems.
However, many researchers did not focus on creating acting AI succeeding in this test.
The following self-explanatory quote from \cite{russel2010} may convince many: ``Aeronautics is not defined as the field of making machines that fly so exactly like pigeons that they can fool even other pigeons''.

This manuscript concerns agents acting rationally but also concerns learning.
We still defer the definition of learning, and hereafter, we present that acting rationally can be achieved without learning.
Indeed, creating an agent to solve Sudoku games is possible by exploiting the game rules.
It will sequentially enumerate all cells' available numbers and assign a number to cells that end with only one possible number.
Many problems can be solved without learning, but some may be challenging.
For example, an agent can play chess by enumerating all the possibilities from a given board situation to choose the best move.
But the number of possible chess games is $10^{120}$ \citep{shannon1950}, and enumerating all of them is intractable.
However, creating an agent that plays chess is possible if it enumerates only the games of interest and does it in parallel to make it fast.
This is how DeepBlue \citep{campbell2002deep} plays chess at the level of the best human players without learning.
Even though it is possible to solve complex problems by ingeniously exploiting game rules, the complexity of some others leads to the development of agents learning to make decisions.

When there is learning in AI, we call it machine learning (ML): ``Machine learning is a subset of AI that learns to make decisions by fitting mathematical models to observed data''~\citep{prince2023understanding}.
ML is divided into different areas based on available data and goals.
For example, an agent can learn to predict the type of a traffic sign based on its picture.
To achieve this, a model can be built with the help of a dataset of pictures and their corresponding labels.
In this case, the model is a mathematical function that maps a picture, a collection of pixel values, to a type of traffic sign.
The model performance is initially poor but improves during training. 
It learns!
During the training phase, the system predicts the types of images from the dataset.
By using actual and predicted types of traffic signs, the model is refined to improve it.
Such a solution is called supervised learning (SL) because the labels supervise the learning.
Another area of ML is unsupervised learning when labels are not provided with the data.
Indeed, the lack of labels induces the absence of supervision, but the model can still learn patterns in the image structure instead of learning to predict its label.
Removing part of the image and training the model to predict the missing pixel is an example of unsupervised learning called inpainting.
Unsupervised learning serves many purposes.
For example, it is common to pre-train a model with unlabeled data, which are typically abundant, to establish a foundation of knowledge before introducing labels, which are often more challenging to acquire, to train the model on a specific task using supervised learning.

At the beginning of this introduction, we presented the decision-making process by highlighting parts based on their impact.
Learning to predict something via supervised learning allows one to make decisions that impact the environment.
Indeed, it is possible to train a model that predicts the strength to apply to your car pedals based on a camera in the car.
We also considered chess.
A collection of chess games would also allow an agent to learn to predict the best move based on the ones seen in the dataset.
Aside from the complexity of acquiring such a dataset, these approaches do not allow AI to learn to make decisions but to learn to replicate the decisions, called imitation learning.
Doesn't this remind you of the acting humanly approach of AI defined previously?
Moreover, one crucial aspect left aside until now is the sequential aspect of this rational decision-making because decisions might impact the future.
Most of the time, as in chess, a sequence of decisions is required to achieve a goal.
We need a systematic approach to train an agent to make sequences of rational decisions.
Such an approach exists and is called reinforcement learning (RL).
This is a third area of machine learning, and most of the time, any ML problem is classified as one of the three approaches presented: supervised, unsupervised or reinforcement learning.

In reinforcement learning, an agent acts and receives a reward after taking action.
The reward is the action's outcome, a numerical feedback.
The agent aims to perform actions that provide the maximum possible accumulated reward.
It learns it by trial and error, typically updating its policy based on the received reward after each action.
Back to our example, when not stopping at a stop sign at a crossroad, you can cause an accident, an excellent example of a situation that would provide a negative reward.
But whether you decide to break or to do nothing, your environment is evolving.
Indeed, in addition to receiving rewards, the agent's environment can evolve due to its action.
We then say that the state of the environment evolves based on the action.
The agent is part of the environment and interacts with it, somehow defined inside and outside of it.
Nevertheless, this dynamic nature of the environment adds a new dimension to the learning process because the agent needs to learn to make a sequence of optimal decisions.
Note that it is possible to design an RL problem without environment evolution.

Sudoku and chess can be solved without learning, but the agent needs to know the game's rules.
In RL, this is not usually the case.
The agent does not know how an action will affect its state nor the reward it will receive.
It just receives the reward after taking an action and learns to maximise it.
One of the challenges of RL is that agents face the exploration-exploitation dilemma.
They must balance trying out new actions to discover potentially better policies (exploration) and sticking to actions that have yielded high rewards in the past (exploitation).
Another common challenge lies in the environment, which can also involve delayed rewards. 
The consequences of an action may not be immediate, typically because a sequence of precise actions is required to achieve something.
This delay makes it more challenging for the agent that has to learn whether a reward is due to a previous action or the last one.
This is referred to as the credit assignment challenge.

Reinforcement learning has applications in various domains.
Recommender system \citep{mcinerney2018explore} is an excellent example of an RL application.
Robotics is another application where an agent learns to control a robot, such as training a robotic hand to solve Rubik's cube \citep{akkaya2019solving}.
Designing chips is a complex sequential decision-making problem that can also be solved with the help of RL \citep{mirhoseini2021graph}. 
RL can be applied in healthcare, typically to decide treatment dosage through time \citep{miotto2018deep}.
Finally, games have been a testbed for reinforcement learning for a long time.
Some significant breakthroughs in RL have been achieved thanks to games, such as attaining human performance in Atari games \citep{Mnih2015}.
They offer the best environment for trials and errors because they are natural simulators.
Indeed, the trial-and-error nature of RL requires that an agent interact with its environment a lot.

We just presented applications of RL where a single agent acts rationally to achieve a predefined goal.
However, back to our first example, when riding your car you are not alone.
Other people are also sharing the roads, and your decisions are also based on them, on their behaviour.
When there are multiple agents in the environment, many things change.
How these agents influence the environment must be considered because a multi-agent system has several types of dynamics.
Typically, agents can take actions simultaneously or sequentially.
For example, everyone makes decisions simultaneously on the road, while chess is a turn-based game in which agents make their decisions one after the other.
Moreover, the number of agents is also different in these two examples.
In chess, there are only two agents, while on the road, there can be a lot.
This illustrates the wide range of problems that a multi-agent system can represent, from a game with two agents to a real-world environment with an unknown number of agents.

Multi-agent reinforcement learning (MARL) extends single-agent reinforcement learning (SARL) when multiple agents learn in the environment.
In such a system, each agent receives a reward and learns to act to maximise its rewards. 
Learning is the crucial differentiation between multi-agent systems and MARL.
Suppose only one agent is learning in an environment composed of several ones.
The other agents then have a stationary policy that does not change over time, and they can be considered part of the environment.
However, when several agents learn, their policies change over time.
From the point of view of each agent, the environment dynamic is constantly changing.
This induces non-stationarity in the learning process, known as a moving target problem.
Typically, in a rock-paper-scissor game, if one always chooses rock, one may always choose paper.
But if the first changes to scissors, the second may finally change to rock.
This cycle can last forever, and one needs to decide when to stop.
Finding an equilibrium where strategies stop evolving is indeed a challenging solution.
This highliths a first challenge in multi-agent RL.

A second challenge is determining the criterion required to assess the optimality of policies.
In SARL, it is straightforward: the agent must achieve a high sum of rewards, and when it finds the optimal policy achieving the maximum sum of rewards, it stops.
But how can this be posed in a multi-agent environment when several agents want to maximise their reward?
The solution of finding an equilibrium is achieved when agents do not benefit from a better reward by changing their policy.
How to define this equilibrium and whether there are several of them is a dedicated research problem in MARL.
Moreover, finding the equilibrium providing the maximum sum of rewards to all agents can be challenging.

The third challenge is the number of agents.
The environment's dynamic is often a function of every agent's action, leading to a combination of actions that scales exponentially with the number of agents.
Designing methods to train these agents that scale is challenging because it usually implies functions based on all these possibilities.

A fourth challenge is related to the credit assignment previously introduced.
In addition to understanding that a reward may be caused by a series of actions, a challenge in MARL is determining the actions of which agents caused the current reward.
Back to the example, if you stop because of a stop sign but get damaged by the one following you who did not stop, the corresponding negative reward is not your fault, but the faulty one is the car's driver behind you.
Other challenges exist, but these four highlight the additional challenges when extending to multi-agent reinforcement learning \cite{marl-book}.

Multi-agent reinforcement learning has been divided into three settings.
This division comes from game theory and allows simplifying the general case to a particular one to reduce the complexity arising from the challenges previously mentioned.
Each setting is based on the goal of agents.
The first setting, cooperation, involves agents cooperating to achieve a common goal.
In the second setting, competition, agents compete to achieve an opposing goal, typically named the zero-sum setting.
The third setting is named general-sum and encompasses everything else.
This is the general setting and maybe the most challenging because no hypothesis can be exploited to ease the problem.

This introduction clearly identifies that we must consider other learning agents when training one in MARL.
The reader should be confident that training an agent with a single-agent reinforcement learning method, ignoring other learning agents, might not yield the best solution to these problems.
This research question has existed in the MARL community for quite some time, and we go beyond by providing insights in this manuscript into how we should consider other learning agents in an environment.
We focus on two specific settings: training a team of agents to cooperate and training a team of agents to compete against an opposing one.
In the cooperative setting, we present two contributions that provide methods designed to train agents to fulfil a cooperative task and how they can be applied in a specific real-world setting.
When training a team in an adversarial scenario, we can use the same methods but, again, by considering that there are opponents.
This highlights the two main parts of the manuscript, each starting with a background section that provides perspective on the novelties of the presented work.
Before diving into the details of these two settings, we provide the necessary background and notations of reinforcement learning.
In the following sections, we provide a more detailed outline of this manuscript, followed by a disclaimer on the publications that form its foundation and, finally, the industrial context in which some of the work presented has been conducted.

\section{Outline}
\label{sec:ch1_outline}
This manuscript is divided into four parts.
The necessary background is presented in Part~\ref{part:background}, providing an overview of reinforcement learning.
In Part \ref{part:coop}, we present works related to cooperation, one of the multi-agent settings, while Part \ref{part:compet} extends these topics to a specific general-sum framework where two teams compete.
Finally, Part \ref{part:conclusion} concludes this manuscript, retrospectively reviewing the contributions presented in this manuscript.

\begin{description}
    \item [Part \ref{part:background}] Background:
    
    Chapter~\ref{ch:background} is the single chapter of Part \ref{part:background} presenting the basic multi-agent reinforcement learning framework and details its different settings and challenges.
    More importantly, this chapter presents the foundations of MARL issued from single-agent reinforcement learning.
    This chapter is a must-read for those unfamiliar with reinforcement learning. 
    If more details are required, we refer the reader to numerous additional works.
    In contrast, these concepts should be well-known for the familiar ones, and Chapter~\ref{ch:background} serves as a notation introduction.

    \item [Part \ref{part:coop}] Learn to cooperate:
    
    Chapter~\ref{ch:cooperation} is the first of this second part and provides the necessary material to understand the challenges of cooperation in multi-agent reinforcement learning.
    It is a mix of background introduction and literature review, as this chapter introduces many concepts and methods from the literature.
    
    In the second chapter, Chapter~\ref{ch:qvmix}, we present four methods based on the Deep-Quality Value family of algorithms designed to train cooperating agents.
    
    Chapter~\ref{ch:impmarl} is the third and last chapter of Part \ref{part:coop} and introduces infrastructure management planning, a real-world application that can be tackled in this cooperative framework.
    In this application, inspections, repairs, and/or retrofits should be timely planned to control the risk of potential system failures.
    
    \item [Part \ref{part:compet}]  Cooperate against an opposing team:
    
    Chapter~\ref{ch:competition} is the first one and presents how competition is classically modelled and solved in multi-agent reinforcement learning.
    It serves the same purpose as Chapter~\ref{ch:cooperation} by being a mix between background and literature review.
    
    Chapter~\ref{ch:2teams} presents a particular case of the general-sum settings where two teams compete and how to train them.

    \item [Part \ref{part:conclusion}] Conlusion:
    
    The last chapter of this manuscript, Chapter~\ref{ch:conclusion}, addresses the closing remarks.
\end{description}

\section{Publications}
\label{sec:ch1_publications}

This manuscript is built upon existing research, integrating insights from numerous cited sources.
Notably, three central chapters represent adaptations of specific peer-reviewed publications.

\begin{itemize}
\item Chapter~\ref{ch:qvmix} is an adapted version of the publication \textit{QVMix and QVMix-Max: extending the deep quality-value family of algorithms to cooperative multi-agent reinforcement learning}, P. Leroy, D. Ernst, P. Geurts, G. Louppe, J. Pisane, and M. Sabatelli. AAAI-21 Workshop on Reinforcement Learning in Games, 2021 \citep{leroy2020qvmix}.

\item Chapter~\ref{ch:impmarl} is an adapted version of the publication \textit{IMP-MARL: a suite of environments for large-scale infrastructure management planning via MARL}, P. Leroy, P. G. Morato, J. Pisane, A. Kolios, and D. Ernst. Thirty-seventh Conference on Neural Information Processing Systems Datasets and Benchmarks Track, 2023 \citep{leroy2023impmarl}.

\item Chapter~\ref{ch:2teams} is an adapted version of the publication \textit{Value-based CTDE methods in symmetric two-team Markov game: from cooperation to team competition}, P. Leroy, J. Pisane, and D. Ernst. Deep Reinforcement Learning Workshop NeurIPS, 2022 \citep{leroy2022twoteam}.
\end{itemize}


\section{Context}
\label{sec:ch1_context}
To provide a backstory, part of the work presented in this manuscript has been developed during the IRIS project with an industrial consortium of Belgian companies.
IRIS stands for Intelligent Recognition Information System, and the partners were John Cockerill Defense, ACIC, Multitel, Belgian Royal Military Academy and the University of Li{\'{e}}ge.
The IRIS project aims to enhance surveillance capabilities by integrating machine learning.
Through detection, recognition, and behaviour analysis modules, IRIS seeks to empower operators with timely, actionable information to make optimal decisions in both military and civilian contexts, impacting the safety of individuals, sensitive sites, and equipment integrity.
Specifically, we explored multi-agent reinforcement learning to provide decision-aid to operators in the context of military missions.
The IRIS project served as a case of our studies on training a team against an opposing team to learn how military assets should cooperate.

\newpage
\section{TODOLIST}

\todo{Uniformiser RL MARL ou reinforcement learning multi-agent reinforcement learning}

% \todo{read Matthia's thesis to add more?}

\todo{Remove we or you when talking about algorithms}

\todo{check overfull box}

\todo{Figure mosaique avec les environments pour que ça soit joli}

\todo{Refaire les schémas des méthodes (tout en noir, enlever mot en mathmode)}

\todo{changer la figure 5.3 pour supperposer au lieu d'aligner (code matplotlib mosaic)}py

\todo{Couleurs et style des chapitres, citations etc}
